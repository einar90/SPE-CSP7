% -*- program: xelatex -*-

\documentclass{beamer}
\usepackage[utf8]{inputenc}
\usepackage{verbatim}
\usepackage{booktabs}


% -*- root: presentation.tex -*-


\title[SPE CSP7]
{ %
	7th SPE Comparative Solution Poject: \\
	Modelling Horizontal Wells in Reservoir Simulation
}
 
\subtitle{Recreation of results and modification of model}
 
\author[Baumann, Einar] % (optional, for multiple authors)
{Einar Baumann}
 
\institute[NTNU] % (optional)
{
  Department of Petroleum Engineering and Applied Geophysics\\
  Norwegian University of Science and Technology
}
 
\date[\today] % (optional)
{\today}
 
%\logo{\includegraphics[height=.5cm]{ntnu.pdf}}% -*- root: presentation.tex -*-
 

\useoutertheme{clean}
\usecolortheme{clean}
\usefonttheme{clean}

\begin{document}
 
\frame{\titlepage}

\begin{frame}
    \frametitle{Table of Contents}
    \tableofcontents
\end{frame}
 
\section{Purpose}
\begin{frame}
    \frametitle{Purpose}
    \begin{itemize}
        \item The purpose of SPE CSP7 study was to compare the results of simulations performed by different organizations, using different simulators on a problem invlolving production from a horizontal well.
        \item The purpose of this project was to recreate the results of the SPE CSP7 study; specifically the ECL100 results.
        \item Additionally, modifications were made to the model and the results compared to that of the original model.
    \end{itemize}
\end{frame}

\section{Original model}
\begin{frame}
    \frametitle{Original model}
\end{frame}

\subsection{Grid}
\begin{frame}
    \frametitle{Original model: Grid}
    \begin{columns}[c]
        \column{0.7\textwidth}
            \includegraphics[width=\textwidth]{figures/grid.pdf}
        \column{0.45\textwidth}
            \includegraphics[width=\textwidth]{figures/grid-side.pdf}
            \begin{itemize}
                \item $9\times 9 \times 6$ blocks.
                \item Constant $\Delta x$.
                \item Variable $\Delta y$ and $\Delta z$. 
                \item Smaller blocks around the producing well.
            \end{itemize}
    \end{columns}
\end{frame}


\subsection{Wells}
\begin{frame}
    \frametitle{Original model: Wells}
    \begin{itemize}
        \item The model has two wells:
            \begin{itemize}
                \item Producer in top layer.
                \item Injector in bottom layer.
            \end{itemize}
        \item There are two variations on the model:
        \begin{itemize}
             \item short producing well spanning blocks $(I,5,1), I=6,7,8$.
             \item long producing well spanning blocks $(I,5,1), I=2,3,\dots,8$.
        \end{itemize}
        \item In both cases the injector spans blocks $(I,5,6),I=1,2,\dots,9$.
    \end{itemize}
\end{frame}

\begin{frame}
    \frametitle{Cases}
    \footnotesize
    \begin{tabular}{llll}
        \toprule
        \textbf{Case} & \textbf{Producer length} & \textbf{Liquid production rate} & \textbf{Injection scheme} \\
        \midrule
        1a            & $900$ ft                   & $3000$                          & $p = 3700$ psia           \\
        1b            & $2100$ ft                  & $3000$                          & $p = 3700$ psia           \\
        2a            & $900$ ft                   & $6000$                          & $p = 3700$ psia           \\
        2b            & $2100$ ft                  & $6000$                          & $p = 3700$ psia           \\
        3a            & $900$ ft                   & $9000$                          & $p = 3700$ psia           \\
        3b            & $2100$ ft                  & $9000$                          & $p = 3700$ psia           \\
        4a            & $900$ ft                   & $9000$                          & $q_w = 6000$ stb/day      \\
        4b            & $2100$ ft                  & $9000$                          & $q_w = 6000$ stb/day      \\
        \bottomrule
    \end{tabular}
    \normalsize
    \begin{itemize}
        \item Cases 1-3 examine the effects of rates and well lengths on recovery.
        \item In case 4 the voidage relpacement ratio is less than unity, leading to a substantial amount of gas comming out of solution.
    \end{itemize}
\end{frame}

% Differences:
% 18.9000     9.1000    32.1000    17.3000    41.8000    22.6000  -512.7000  -739.9000
% 30.8000    42.5000    37.8000    66.0000    38.1000    75.4000  -595.3000  -740.3000
%
% Deviations:
% 0.0249604   0.0095689   0.0310385   0.0138289   0.0340086   0.0156423  -0.7358978  -0.8942470
% 0.0406762   0.0446898   0.0365500   0.0527578   0.0309983   0.0521872  -0.8544567  -0.8947305
\begin{frame}
    \frametitle{Deviations}
    Total oil production in MSTB after 1500 days.

    \vspace{2em}
    
    \footnotesize
    \begin{columns}[c]
        \column{1.2\textwidth}
        \begin{tabular}{lllllllll}
         \toprule
         \textbf{Case}        & 1a     & 1b     & 2a     & 2b     & 3a     & 3b     & 4a     & 4b  \\
         \midrule
         \textbf{Paper (ECL)} & 757.2  & 951.0  & 1034.2 & 1251.0 & 1229.1 & 1444.8 & 696.7  & 827.4  \\
         \midrule
         \textbf{Orig. grid}  & 738.3  & 941.9  & 1002.1 & 1233.7 & 1187.3 & 1422.2 & 1209.4 & 1567.3 \\
         Deviation            & 2.5\%  & 1.0\%  & 3.1\%  & 1.4\%  & 3.4\%  & 1.6\%  & -73.6\%& -89.4\%\\
         \midrule
         \textbf{Ref. grid}   & 726.4  & 908.5  & 996.4  & 1185.0 & 1191.0 & 1369.4 & 1292.0 & 1567.7 \\
         Deviation            & 4.1\%  & 4.5\%  & 3.7\%  & 5.3\%  & 3.1\%  & 5.2\%  & -85.4\%& -89.5\%\\
         \bottomrule
        \end{tabular}
    \end{columns}
\end{frame}

\begin{frame}[fragile]
	\frametitle{Sample frame title}
	\begin{code}
        \begin{verbatim}
        this is code
is it indented?
 nope...
this is code
is it indented?
 nope...
        \end{verbatim}
    \end{code}
\end{frame}

\begin{frame}
 In this slide \pause
 
 the text will be partially visible
\end{frame}


\begin{frame}
	\begin{remark}
		Sample text in green box.
	\end{remark}
\end{frame}
 
\end{document}
