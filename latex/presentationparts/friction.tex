% -*- root: ../presentation.tex -*-
\section{Friction}

\subsection{Eclipse implementation}
%==============================================================================
\begin{frame}[fragile]
    \frametitle{Friction}
    \begin{itemize}
        \item Friction in the producing well is accounted for using the \texttt{WFRICTN}-keyword:
    \end{itemize}

\vspace{1em}

    \begin{code}
        \begin{verbatim}
WFRICTN
-- well name  tubing diam  roughness  scale
   PROD       0.375        10.0E-3    1.0 /
-- I  J  K      Direction  RangeEnd   Tubing diam
   8  5  1  2*  X          2          1* /
/
        \end{verbatim}
    \end{code}

    \begin{itemize}
        \item Enables calculation of pressure losses between connections and the well's bottom hole reference point.
    \end{itemize}

\end{frame}
%==============================================================================


\subsection{Model}
%==============================================================================
\begin{frame}
    \frametitle{Friction: Model}
    According to R.A. Archer \& E.O Agbongiator\footnote{R.A. Archer, E.O. Agbongiator. \emph{Correcting for Frictional Pressure Drop in Horizontal-Well Inflow-Performance Relationships}. Society of Petroleum Engineers, 2005}, wellbore pressure drops in horizontal wells may be modelled with equation \ref{eq:friction}.
    \begin{equation}
        \label{eq:friction}
        \Delta p_f = \frac{C_f L f \rho q_f^2}{D^5}
    \end{equation}
    where
    \begin{description}
        \item[$C_f=$] unit conversion factor, 
        \item[$f=$] Fanning friction factor,
        \item[$L=$] length of productive horizontal interval, ft,
        \item[$\rho=$] density of produced fluid, lb/ft3, and
        \item[$D=$] wellbore diameter, ft.
    \end{description}
\end{frame}
%==============================================================================



%==============================================================================
\begin{frame}
    \frametitle{Friction: Model (2)}
    \begin{itemize}
            \item Pressure drop increases with the square of the flow rate.
            \item Flow rate ``accumulates'' along the length of the wellbore towards the heel.
            \begin{itemize}
                \item The flow rate will be higher at the heel.
                \item The pressure drop will be higher at the heel.
            \end{itemize}
        \end{itemize}    
\end{frame}
%==============================================================================


\subsection{Pressure drop along wellbore}
%==============================================================================
\begin{frame}
    \frametitle{Friction: Pressure Drop Along Wellbore}
    \centerline{% -*- root: ../../../plots.tex -*-

\begin{tikzpicture}
\pgfplotsset{legend style={at={(0.5,-0.30)}, font=\footnotesize}}
\pgfplotsset{width=1.0\textwidth, height=0.7\textheight}
        \begin{axis}[
        xlabel={Connection $(I,4,5)$},
        ylabel={Pressure Drop (psia)},
        xtick={1,2,...,8},
        ]
        \addplot[color=red, mark=o] table[x=CONN, y=1A:D1:DROP, col sep=comma, skip coords between index={0}{4}]{data/cpr.csv};
        \addlegendentry{1a, D1};
        \addplot[color=green, mark=o] table[x=CONN, y=1A:D700:DROP, col sep=comma, skip coords between index={0}{4}]{data/cpr.csv};
        \addlegendentry{1a, D700};
        \addplot[color=blue, mark=o] table[x=CONN, y=1A:D1500:DROP, col sep=comma, skip coords between index={0}{4}]{data/cpr.csv};
        \addlegendentry{1a, D1500};

        \addplot[color=red, mark=x] table[x=CONN, y=1B:D1:DROP, col sep=comma]{data/cpr.csv};
        \addlegendentry{1b, D1};
        \addplot[color=green, mark=x] table[x=CONN, y=1B:D700:DROP, col sep=comma]{data/cpr.csv};
        \addlegendentry{1b, D700};
        \addplot[color=blue, mark=x] table[x=CONN, y=1B:D1500:DROP, col sep=comma]{data/cpr.csv};
        \addlegendentry{1b, D1500};
    \end{axis}
\end{tikzpicture}}
\end{frame}
%==============================================================================


%==============================================================================
\begin{frame}
    \frametitle{Friction: Pressure Drop Along Wellbore}
    \centerline{% -*- root: ../../../plots.tex -*-
\pgfplotsset{legend style={legend columns=2}}
\begin{tikzpicture}
\pgfplotsset{legend style={at={(0.5,-0.30)}, font=\footnotesize}}
\pgfplotsset{width=1.0\textwidth, height=0.7\textheight}
        \begin{axis}[
        xlabel={Connection $(I,4,5)$},
        ylabel={Pressure Drop (psia)},
        xtick={1,2,...,8},
        ]
        \addplot[color=red, mark=o] table[x=CONN, y=1B:D1:DROP, col sep=comma]{data/cpr.csv};
        \addlegendentry{1b: $q=3000$};
        \addplot[color=green, mark=o] table[x=CONN, y=2B:D1:DROP, col sep=comma]{data/cpr.csv};
        \addlegendentry{2b: $q=6000$};
        \addplot[color=blue, mark=o] table[x=CONN, y=3B:D1:DROP, col sep=comma]{data/cpr.csv};
        \addlegendentry{3b: $q=9000$};
        \addplot[color=gray, mark=o] table[x=CONN, y=4B:D1:DROP, col sep=comma]{data/cpr.csv};
        \addlegendentry{4b: $q=9000$};
    \end{axis}
\end{tikzpicture}}
\end{frame}
%==============================================================================

%==============================================================================
\begin{frame}
    \frametitle{Friction: Pressure Drop Along Wellbore}
    \centerline{% -*- root: ../../../plots.tex -*-
\pgfplotsset{legend style={legend columns=2}}
\begin{tikzpicture}
\pgfplotsset{legend style={at={(0.5,-0.30)}, font=\footnotesize}}
\pgfplotsset{width=1.0\textwidth, height=0.7\textheight}
        \begin{axis}[
        xlabel={Connection $(I,4,5)$},
        ylabel={Pressure Drop (psia)},
        ymin=0,
        xtick={1,2,...,8},
        ]
        \addplot[color=red, mark=o] table[x=CONN, y=1A:D1:DROP, col sep=comma, skip coords between index={0}{4}]{data/cpr.csv};
        \addlegendentry{1a: $q=3000$};
        \addplot[color=green, mark=o] table[x=CONN, y=2A:D1:DROP, col sep=comma, skip coords between index={0}{4}]{data/cpr.csv};
        \addlegendentry{2a: $q=6000$};
        \addplot[color=blue, mark=o] table[x=CONN, y=3A:D1:DROP, col sep=comma, skip coords between index={0}{4}]{data/cpr.csv};
        \addlegendentry{3a: $q=9000$};
        \addplot[color=gray, mark=o] table[x=CONN, y=4A:D1:DROP, col sep=comma, skip coords between index={0}{4}]{data/cpr.csv};
        \addlegendentry{4a: $q=9000$};
    \end{axis}
\end{tikzpicture}}
\end{frame}
%==============================================================================

%==============================================================================
\begin{frame}
    \frametitle{Friction: Pressure Drop Along Wellbore}
    \centerline{% -*- root: ../../../plots.tex -*-

\begin{tikzpicture}
\pgfplotsset{legend style={legend columns=2}}
\pgfplotsset{legend style={at={(0.5,-0.30)}, font=\footnotesize}}
\pgfplotsset{width=1.0\textwidth, height=0.7\textheight}
        \begin{axis}[
        xlabel={Connection $(I,4,5)$},
        ylabel={Pressure Drop (psia)},
        xtick={1,2,...,8},
        ]
        \addplot[color=red, mark=o] table[x=CONN, y=1A:D1:DROP, col sep=comma, skip coords between index={0}{4}]{data/cpr.csv};
        \addlegendentry{1a: Original grid};

        \addplot[color=red, mark=o, dashed] table[x=CONN, y=REF:1A:D1:DROP, col sep=comma, skip coords between index={0}{4}]{data/cpr.csv};
        \addlegendentry{1a: Refined grid};

        \addplot[color=blue, mark=o] table[x=CONN, y=1B:D1:DROP, col sep=comma]{data/cpr.csv};
        \addlegendentry{1b: Original grid};

        \addplot[color=blue, mark=o, dashed] table[x=CONN, y=REF:1B:D1:DROP, col sep=comma]{data/cpr.csv};
        \addlegendentry{1b: Refined grid};
    \end{axis}
\end{tikzpicture}}
\end{frame}
\begin{frame}
    \begin{itemize}
        \item The refined model has more blocks in the $I$-direction.
        \item The second block and the second-to-last block were dropped from the data set to make it ``compatible'' with the original data set.
        \begin{itemize}
            \item This explains why the pressure drop in the eigth block is higher in the refined model: the block called the eigth block in the refined data set is actually block 11, which is closer to the bottom hole reference point.
        \end{itemize}
    \end{itemize}
\end{frame}
%==============================================================================
